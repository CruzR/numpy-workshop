\documentclass{beamer}

\usepackage[utf8]{inputenc}
\usepackage[ngerman]{babel}
\usepackage{graphicx}

\title{Datenbasierte Entscheidungsfindung mit Python}
\author{Thomas Glamsch}
\date{22.~April~2017}

\begin{document}

\begin{frame}
  \titlepage
\end{frame}

\section{Motivation}

\begin{frame}
  \frametitle{Problemstellung}
  Anhand von
  \begin{itemize}
  \item mathematischen Modellen,
  \item Simulationen oder
  \item statistischen Daten
  \end{itemize}
  soll eine fundierte Entscheidung getroffen werden.

  \pause

  \begin{block}{Kontext}
    Viele verschiedene Kontexte sind denkbar, z.B.
    \begin{itemize}
    \item in der Arbeit,
    \item im Sportverein,
    \item bei der privaten Investitionsplanung
    \end{itemize}
  \end{block}
\end{frame}

\begin{frame}
  \frametitle{Konkrete Beispiele}
  \begin{itemize}
  \item Angebote verschiedener Online-Broker vergleichen, \\ um das am besten auf die pers\"onlichen Bed\"urfnisse zugeschnittene zu finden
  \item Rundungsfehler beim Anlegen einer Sinus-Tabelle \\ f\"ur ein embedded-Projekt bestimmen
  \item Erwarteten Wiederverkaufswert eines Autos berechnen, \\ um den besten Zeitpunkt f\"ur einen Verkauf zu finden
  \item Jobangebote in verschiedenen St\"adten unter Einbeziehung \\ der Lebenshaltungskosten vergleichen
  \end{itemize}
\end{frame}

\begin{frame}
  \frametitle{Tabellenkalkulationen}
  \includegraphics[width=4in]{libreoffice-logo.png}
\end{frame}

\begin{frame}
  \frametitle{Probleme mit Tabellenkalkulationen}
  \begin{block}{Expressivit\"at}
  \note{bestimmte Fragestellungen schwierig \\ bzw. umst\"andlich zu beantworten}
    \begin{itemize}
    \item Statistiken \"uber Kategorien
    \item sich \"andernde Anzahl von Datenpunkten
    \end{itemize}
  \end{block}
\end{frame}

\begin{frame}
  \frametitle{Probleme mit Tabellenkalkulationen}
  \begin{block}{Korrektheit}
    \note{schwer auf Korrektheit zu \"uberpr\"ufen}
    \begin{itemize}
    \item Woher stammen die Daten?
    \item Wurden sie korrekt eingegeben?
    \item Sind alle Formeln korrekt?
    \end{itemize}
  \end{block}
\end{frame}

\begin{frame}
  \frametitle{Probleme mit Tabellenkalkulationen}
  \begin{block}{Nachvollziehbarkeit}
    \note{Probleme mit der Nachvollziehbarkeit}
    \begin{itemize}
    \item Warum wurde welche Metrik berechnet?
    \item Was f\"ur Schritte wurden unternommen, \\ um zum Endergebnis zu gelangen?
    \item Anhand welcher Kriterien wurde die Entscheidung gef\"allt?
    \item Welche Entscheidung wurde gef\"allt?
    \end{itemize}
  \end{block}
\end{frame}


\section{Einf\"uhrung}

\subsection{Jupyter}

\begin{frame}
  \frametitle{Jupyter Notebooks}
  \includegraphics[width=2in]{jupyter-logo.png}
\end{frame}

\subsection{Python}

\begin{frame}
  \frametitle{Programmiersprache Python}
  \includegraphics[width=2in]{python-logo.png}
\end{frame}

\begin{frame}
  \frametitle{Python als Taschenrechner}
\end{frame}


\subsection{NumPy}

\begin{frame}
  \frametitle{Mathematische Berechnungen mit NumPy}
  \includegraphics[width=0.8in]{numpy-logo.png}
\end{frame}

\begin{frame}
  \frametitle{NumPy-Arrays}
\end{frame}


\subsection{Matplotlib}

\begin{frame}
  \frametitle{Diagramme mit Matplotlib}
  \includegraphics[width=1.8in]{matplotlib-logo.png}
\end{frame}

\section{Installation}

\begin{frame}
  \frametitle{Installation der Distributions-Pakete}
  Wir ben\"otigen Python in der Version 3.4 oder neuer sowie das Python-Modul \emph{venv}.
  Die genauen Paketnamen h\"angen von der jeweiligen Linux-Distribution ab.
  Unter Debian Jessie etwa installiert man die Pakete mit:
  \begin{semiverbatim}
    \# apt-get install python3 python3-venv
  \end{semiverbatim}
  Um erstellte Dokumente als PDF zu exportieren \\ wird XeLaTeX gebraucht:
  \begin{semiverbatim}
    \# apt-get install texlive texlive-xetex
  \end{semiverbatim}
\end{frame}

\begin{frame}
  \frametitle{Aufsetzen einer virtuellen Python-Umgebung}
  Um sich nicht mit anderen Python-Programmen in die Quere zu kommen,
  macht es Sinn, alle ben\"otigten Python-Pakete in ein lokales Verzeichnis
  zu installieren.
  Dazu setzen wir eine sogenannte \emph{virtualenv}, also eine virtuelle
  Python-Umgebung auf:
  \begin{semiverbatim}
    \$ python3 -m venv \$VENV\_DIRECTORY
  \end{semiverbatim}
  Damit Befehle wie \emph{pip} auf diesem Verzeichnis arbeiten, \\
  m\"ussen wir einige Umgebungsvariablen einlesen:
  \begin{semiverbatim}
    \$ source \$VENV\_DIRECTORY/bin/activate
  \end{semiverbatim}
\end{frame}

\begin{frame}
  \frametitle{Installation der Python-Pakete}
  Wir aktualisieren zun\"achst den Python-Paketmanager pip,
  da veraltete Versionen unter Umst\"anden die Abh\"angigkeiten
  der zu installierenden Pakete nicht korrekt aufl\"osen:
  \begin{semiverbatim}
    \$ pip3 install --upgrade pip
  \end{semiverbatim}
  Dann installieren wir Jupyter, NumPy und matplotlib:
  \begin{semiverbatim}
    \$ pip3 install jupyter numpy matplotlib
  \end{semiverbatim}
\end{frame}

\begin{frame}
  \frametitle{Beispiel: Jobangebote vergleichen}
  Was muss alles mit einbezogen werden?
  \begin{itemize}
  \item Gehalt (inkl. W\"ahrung)
  \item Nichtmonet\"are Gehaltskomponenten
    \begin{itemize}
    \item Unternehmensanteile (\emph{equity})
    \item Betriebsrenten
    \end{itemize}
  \item Mieten
  \item Lebensmittelpreise
  \item Steuern
  \item Gesundheits-/Sozialabgaben
  \item Sonstige Faktoren (schwer zu quantifizieren, aber oft ausschlaggebend)
    \begin{itemize}
    \item \emph{Cultural Fit} mit dem Unternehmen
    \item Umzug n\"otig?
    \item Umzug m\"oglich? (Familie, Freunde, ...)
    \end{itemize}
  \end{itemize}
\end{frame}

\begin{frame}
  \frametitle{Ideen f\"ur den Vortrag}
  \begin{itemize}
  \item Wie kann man bestimmen, welche Faktoren einflie{\ss}en sollen?
    \begin{itemize}
    \item Dom\"anenwissen notwendig
    \item Vorher strukturiert Gedanken machen
    \end{itemize}
  \item Erkenntnisse aus Psychologie / Verhaltens\"okonomik (\emph{Behavioral Economics}) dazu, wie Menschen Entscheidungen treffen
  \end{itemize}
\end{frame}
\end{document}