\documentclass{beamer}

\usepackage[utf8]{inputenc}
\usepackage[ngerman]{babel}
\usepackage{graphicx}

\title{Datenbasierte Entscheidungsfindung mit Python}
\author{Thomas Glamsch}
\date{22.~April~2017}

\begin{document}

\begin{frame}
  \titlepage
\end{frame}

\section{Motivation}

\begin{frame}
  \frametitle{Problemstellung}
  Anhand von
  \begin{itemize}
  \item mathematischen Modellen,
  \item Simulationen oder
  \item statistischen Daten
  \end{itemize}
  soll eine fundierte Entscheidung getroffen werden.

  \pause

  \begin{block}{Kontext}
    Viele verschiedene Kontexte sind denkbar, z.B.
    \begin{itemize}
    \item in der Arbeit,
    \item im Sportverein,
    \item bei der privaten Investitionsplanung
    \end{itemize}
  \end{block}
\end{frame}

\begin{frame}
  \frametitle{Tabellenkalkulationen}
  \includegraphics[width=4in]{libreoffice-logo.png}
\end{frame}

\begin{frame}
  \frametitle{Probleme mit Tabellenkalkulationen}
  \begin{block}{Expressivit\"at}
  \note{bestimmte Fragestellungen schwierig \\ bzw. umst\"andlich zu beantworten}
    \begin{itemize}
    \item Statistiken \"uber Kategorien
    \item sich \"andernde Anzahl von Datenpunkten
    \end{itemize}
  \end{block}
\end{frame}

\begin{frame}
  \frametitle{Probleme mit Tabellenkalkulationen}
  \begin{block}{Korrektheit}
    \note{schwer auf Korrektheit zu \"uberpr\"ufen}
    \begin{itemize}
    \item Woher stammen die Daten?
    \item Wurden sie korrekt eingegeben?
    \item Sind alle Formeln korrekt?
    \end{itemize}
  \end{block}
\end{frame}

\begin{frame}
  \frametitle{Probleme mit Tabellenkalkulationen}
  \begin{block}{Nachvollziehbarkeit}
    \note{Probleme mit der Nachvollziehbarkeit}
    \begin{itemize}
    \item Warum wurde welche Metrik berechnet?
    \item Was f\"ur Schritte wurden unternommen, \\ um zum Endergebnis zu gelangen?
    \item Anhand welcher Kriterien wurde die Entscheidung gef\"allt?
    \item Welche Entscheidung wurde gef\"allt?
    \end{itemize}
  \end{block}
\end{frame}


\section{Einf\"uhrung}

\subsection{Jupyter}

\begin{frame}
  \frametitle{Jupyter Notebooks}
  \includegraphics[width=2in]{jupyter-logo.png}
\end{frame}

\subsection{Python}

\begin{frame}
  \frametitle{Programmiersprache Python}
  \includegraphics[width=2in]{python-logo.png}
\end{frame}

\begin{frame}
  \frametitle{Python als Taschenrechner}
\end{frame}


\subsection{NumPy}

\begin{frame}
  \frametitle{Mathematische Berechnungen mit NumPy}
  \includegraphics[width=0.8in]{numpy-logo.png}
\end{frame}

\begin{frame}
  \frametitle{NumPy-Arrays}
\end{frame}


\subsection{Matplotlib}

\begin{frame}
  \frametitle{Diagramme mit Matplotlib}
  \includegraphics[width=1.8in]{matplotlib-logo.png}
\end{frame}

\end{document}